\documentclass[twoside, 11pt]{article}

\usepackage[hmarginratio=1:1,top=8mm, left=8mm, right=8mm,columnsep=10pt]{geometry} % Document margins
\usepackage{amsmath}
\usepackage{amssymb}
\usepackage{multicol}
\usepackage{dsfont} % For \mathds commands

\title{Cryptography and Architectures for Computer Security - Cheat Sheet}
\author{}
\date{}


\begin{document}
    \maketitle
    %\begin{multicols*}{1}
        
        %--- Begin Information Theory
        \subsection*{Information Theory}
            $Pr(C=c) = \sum_{k : c \in \lbrace \mathds{E}_{k}(m), \forall m \in \mathcal{M} \rbrace} Pr(K = k)Pr(P = \mathds{D}_k(c))$ \\
            Perfect secrecy: $Pr(P=m|C=c) = Pr(P=m) \implies |\mathcal{K}| = |\mathcal{C}| = |\mathcal{P}|$ \\
            Entropy: $H(X) = -\sum_{i=1}^{n} p_{i}\log_{2}p_{i}$ \quad $(p_{i}\log_{2}p_{i} = 0$ for $p_{i} = 0)$ \\
            $H(X, Y) = - \sum_{i=1}^{n} \sum_{j=1}^{m}Pr(X = x_{i}, Y = y_{j})\log_{2}Pr(X = x_{i}, Y = y_{j})$ \\
            $H(X | Y = y) = - \sum_{i=1}^{n} \sum_{j=1}^{m}Pr(X=x_{i} |Y = y)\log_{2}Pr(X = x_{i} | Y = y)$ \\
            $H(X | Y) = - \sum_{i=1}^{n} \sum_{j=1}^{m}Pr(Y = y_{j})Pr(X = x_{i}, Y=y_{j})\log_{2}Pr(X = x_{i} | Y = y_{j})$ \\
            $H(X) + H(Y) \geqslant H(X, Y)$; $H(X, Y)  = H(Y) + H(X|Y)$; $H(X|Y) \leqslant H(X)$ \\
            Key equivocation: $H(K|C) = H(P)+H(K)-H(C)$ \\
            Language redundancy: $R_{L} = 1 - \frac{H_{L}}{\log_{2}|\mathcal{M}|}$ \\
            Spurious keys: $\bar{s_{n}} \geqslant \frac{|\mathcal{K}|}{|\mathcal{M}|^{nR_{L}}}-1$ \\
            Unicity distance: $n_{0} \approx \frac{\log_{2}|\mathcal{K}|}{R_{L}\log_{2}|\mathcal{M}|}$
        %--- End Information Theory   
        
        %--- Begin Symmetric Ciphers
        \subsection*{Symmetric Ciphers}
            \subsubsection*{Modes of Operation}
                ECB: $c_{i} = \mathds{E}_{k}(m_{i})$ \\
                CBC: $c_{0}=IV$, $c_{i}=\mathds{E}_{k}(m_{i} \oplus c_{i-1})$ \\
                CFB/OFB: $ISR_{0}=IV$, $OSR_{i} = \mathds{E}_{k}(ISR_{i-1})$, $c_{i} = m_{i} \oplus$ j-th leftmost bits of $OSR_{i}$ \\ 
                CTR: $ctr_{i}=IV+i$, $t_{i} = \mathds{E}_{k}(ctr_{i})$, $c_{i} = t_{i} \oplus m_{i}$
            
            \subsubsection*{Cryptanalysis}
                Pile-up lemma: $Pr(Z_{1} \oplus \cdots \oplus Z_{n} = 0) = \frac{1}{2} + 2^{n-1}\prod_{i=1}^{n}\varepsilon_{i}$ \\

        %--- End Symmetric Ciphers

        %--- Begin Hash Functions
        \subsection*{Hash Functions}
            First preimage: $Pr(m_{i} | d=h(m_{i})) \approx \frac{q}{|D|}$ \\
            Second preimage: $Pr(h(m_{i}) = h(m)) = \frac{q-1}{|D|}$ \\
            Collision: $Pr(\text{no collisions}) = e^{-\frac{q(q-1)}{2|D|}} \implies q \leqslant 1.774\sqrt{|D|}$
        %--- End Hash Functions

        %--- Begin Algebraic Structures
        \subsection*{Algebraic Structures}
        %--- End Algebraic Structures

        %--- Begin Elliptic Curves
        \subsection*{Elliptic Curves}
        %--- End Elliptic Curves

        %--- Begin Public Key Cryptosystems
        \subsection*{Public Key Cryptosystems}
            \subsubsection*{RSA}
                RSA keys: $k_{pub} = (n, e)$, $k_{priv}=(p, q, \varphi(n), d)$ \\
                $n=p\cdot q$, $gcd(e, \varphi(n))=1$, $d=e^{-1}\mod{\varphi(n)}$ \\
                $c=m^{e \mod{\varphi(n)}} \mod{n}$, $m=c^{d \mod{\varphi(n)}} \mod{n}$ \\
                CRT: $m_{p} \equiv_{p} c^{d \mod{p-1}}$, $m_{q} \equiv_{q} c^{d \mod{q-1}}$, $m \equiv_{n} m_{p}q(q^{-1}\mod{p}) + m_{q}p(p^{-1}\mod{p})$
        %--- End Public Key Cryptosystems

        %--- Begin Montgomery Multiplication
        \subsection*{Montgomery Multiplication}
        %--- End Montgomery Multiplication

        %--- Begin Number Theoretical Cryptanalysis
        \subsection*{Number Theoretical Cryptanalysis}
            \subsubsection*{Primality test}
                Fermat: $n$ is composite $\implies a^{n-1} \not\equiv_{n} 1$ with probability $> \frac{1}{2}$ \\ 
                Miller-Rabin: $n-1=d2^{s}: a^{d} \not\equiv_{n} \pm 1$ and $a^{d2^{r}} \not\equiv -1 \implies n$ is composite \\
            \subsubsection*{Factoring}
                Fermat: $x = \lceil \sqrt{n} \rceil, y=x^{2}-n$, until y is a perfect square $y=y+2x+1, x = x+1$, then the factors are $x \pm \sqrt{y}$ \\
                Pollard's $\rho$: pick $a, b$ at random (e.g $x_{0}=2, x_{i}=x_{i-1}^{2} \mod{n})$), if $gcd(a-b, n) \neq 1$ the result is a factor \\
                Pollard's $p-1$: $p$ B-power-smooth, $a=2^{B!}$, so $p=gcd(a-1, n)$
            \subsubsection*{DLog}
                Polig-Hellman: for each prime factor $\eta = g^{\frac{n}{p}}$, $\gamma_{i}=\gamma_{i-1}g^{l_{i-1}p^{i-1}}$ ,$\delta_{i}=(\beta\gamma_{i}^{-1})^{\frac{n}{p^{i+1}}}$, $l_{i}=\log_{\eta}\delta_{i}$
        %--- End Number Theoretical Cryptanalysis

        %--- Begin Misc
        \subsection*{Misc}
        %--- End Misc
    %\end{multicols*}
\end{document}